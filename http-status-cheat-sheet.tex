\documentclass[10pt,landscape]{article}
\usepackage{garamondx}
\usepackage{courier}
\usepackage[utf8]{inputenc}
\usepackage[english]{babel}
\usepackage{multicol}
\usepackage{calc}
\usepackage{ifthen}
\usepackage[landscape]{geometry}
\usepackage{hyperref}
\usepackage{listings}

% To make this come out properly in landscape mode, do one of the following
% 1.
%  pdflatex latexsheet.tex
%
% 2.
%  latex latexsheet.tex
%  dvips -P pdf  -t landscape latexsheet.dvi
%  ps2pdf latexsheet.ps


% If you're reading this, be prepared for confusion.  Making this was
% a learning experience for me, and it shows.  Much of the placement
% was hacked in; if you make it better, let me know...


% 2008-04
% Changed page margin code to use the geometry package. Also added code for
% conditional page margins, depending on paper size. Thanks to Uwe Ziegenhagen
% for the suggestions.

% 2006-08
% Made changes based on suggestions from Gene Cooperman. <gene at ccs.neu.edu>


% To Do:
% \listoffigures \listoftables
% \setcounter{secnumdepth}{0}


% This sets page margins to .5 inch if using letter paper, and to 1cm
% if using A4 paper. (This probably isn't strictly necessary.)
% If using another size paper, use default 1cm margins.
\ifthenelse{\lengthtest { \paperwidth = 11in}}
	{ \geometry{top=.5in,left=.5in,right=.5in,bottom=.5in} }
	{\ifthenelse{ \lengthtest{ \paperwidth = 297mm}}
		{\geometry{top=1cm,left=1cm,right=1cm,bottom=1cm} }
		{\geometry{top=1cm,left=1cm,right=1cm,bottom=1cm} }
	}

% Turn off header and footer
\pagestyle{empty}
 

% Redefine section commands to use less space
\makeatletter
\renewcommand{\section}{\@startsection{section}{1}{0mm}%
                                {-1ex plus -.5ex minus -.2ex}%
                                {0.5ex plus .2ex}%x
                                {\normalfont\large\bfseries}}
\renewcommand{\subsection}{\@startsection{subsection}{2}{0mm}%
                                {-1explus -.5ex minus -.2ex}%
                                {0.5ex plus .2ex}%
                                {\normalfont\normalsize\bfseries}}
\renewcommand{\subsubsection}{\@startsection{subsubsection}{3}{0mm}%
                                {-1ex plus -.5ex minus -.2ex}%
                                {1ex plus .2ex}%
                                {\normalfont\small\bfseries}}
\makeatother

% Define BibTeX command
\def\BibTeX{{\rm B\kern-.05em{\sc i\kern-.025em b}\kern-.08em
    T\kern-.1667em\lower.7ex\hbox{E}\kern-.125emX}}

% Don't print section numbers
\setcounter{secnumdepth}{0}


\setlength{\parindent}{0pt}
\setlength{\parskip}{0pt plus 0.5ex}

\def\emdash
		{\discretionary{}{}{\kern 0.1em}---\discretionary{}{}{\kern 0.1em}}

\lstset{basicstyle=\ttfamily,
  showstringspaces=false,
}

% -----------------------------------------------------------------------

\begin{document}

\raggedright
\footnotesize
\begin{multicols}{3}


% multicol parameters
% These lengths are set only within the two main columns
%\setlength{\columnseprule}{0.25pt}
\setlength{\premulticols}{1pt}
\setlength{\postmulticols}{1pt}
\setlength{\multicolsep}{1pt}
\setlength{\columnsep}{2pt}
\newlength{\MyLen}

\begin{center}
     \Large{\textbf{\texttt{HTTP} Status Cheat Sheet}} \\
\end{center}

To get the status of a webpage, you can simply run
\begin{lstlisting}[language=bash,numbers=none]
curl -s -o /dev/null -w "%{http_code}" <URL>
\end{lstlisting}

\section{\texttt{1XX}\emdash: Informational}
%\begin{tabular}{@{}ll@{}}
%\verb!100!	& ksjfdnj\\
%\end{tabular}

Request received, continuing process.\\
This class of status code indicates a provisional response, consisting only of the Status-Line and optional headers, and is terminated by an empty line. Since \verb!HTTP/1.0! did not define any \verb!1XX! status codes, servers must not send a \verb!1XX! response to an \verb!HTTP/1.0! client except under experimental conditions.


\subsection{\texttt{100}: Continue}
The server has received the request headers and the client should proceed to send the request body (in the case of a request for which a body needs to be sent; for example, a \verb!POST! request). Sending a large request body to a server after a request has been rejected for inappropriate headers would be inefficient. To have a server check the request's headers, a client must send Expect: \verb!100!-continue as a header in its initial request and receive a \verb!100! Continue status code in response before sending the body. The response \verb!417! Expectation Failed indicates the request should not be continued.


\subsection{\texttt{101}: Switching Protocols}
The requester has asked the server to switch protocols and the server has agreed to do so.

\subsection{\texttt{102}: Processing (WebDAV; RFC 2518)}
A WebDAV request may contain many sub-requests involving file operations, requiring a long time to complete the request. This code indicates that the server has received and is processing the request, but no response is available yet. This prevents the client from timing out and assuming the request was lost.




\section{\texttt{2XX}\emdash: Success}
This class of status codes indicates the action requested by the client was received, understood, accepted, and processed successfully.

\subsection{\texttt{200}: OK}
Standard response for successful \verb!HTTP! requests. The actual response will depend on the request method used. In a \verb!GET! request, the response will contain an entity corresponding to the requested resource. In a \verb!POST! request, the response will contain an entity describing or containing the result of the action.

\subsection{\texttt{201}: Created}
The request has been fulfilled, resulting in the creation of a new resource.

\subsection{\texttt{202}: Accepted}
The request has been accepted for processing, but the processing has not been completed. The request might or might not be eventually acted upon, and may be disallowed when processing occurs.

\subsection{\texttt{203}: Non-Authoritative Information (since \texttt{HTTP/1.1})}
The server is a transforming proxy (e.g. a Web accelerator) that received a \texttt{200} OK from its origin, but is returning a modified version of the origin's response.

\subsection{\texttt{204}: No Content}
The server successfully processed the request and is not returning any content.

\subsection{\texttt{205}: Reset Content}
The server successfully processed the request, but is not returning any content. Unlike a \verb!204! response, this response requires that the requester reset the document view.

\subsection{\texttt{206}: Partial Content (RFC 7233)}
The server is delivering only part of the resource (byte serving) due to a range header sent by the client. The range header is used by \verb!HTTP! clients to enable resuming of interrupted downloads, or split a download into multiple simultaneous streams.

\subsection{\texttt{207}: Multi-Status (WebDAV; RFC 4918)}
The message body that follows is an \texttt{XML} message and can contain a number of separate response codes, depending on how many sub-requests were made.

\subsection{\texttt{208}: Already Reported (WebDAV; RFC 5842)}
The members of a \texttt{DAV} binding have already been enumerated in a previous reply to this request, and are not being included again.

\subsection{\texttt{226}: IM Used (RFC 3229)}
The server has fulfilled a request for the resource, and the response is a representation of the result of one or more instance-manipulations applied to the current instance.



\section{\texttt{3XX}\emdash: Redirection}
This class of status code indicates the client must take additional action to complete the request. Many of these status codes are used in \verb!URL! redirection.

A user agent may carry out the additional action with no user interaction only if the method used in the second request is \verb!GET! or \verb!HEAD!. A user agent may automatically redirect a request. A user agent should detect and intervene to prevent cyclical redirects.

\subsection{\texttt{300}: Multiple Choices}
Indicates multiple options for the resource from which the client may choose. For example, this code could be used to present multiple video format options, to list files with different extensions, or to suggest word sense disambiguation.

\subsection{\texttt{301}: Moved Permanently}
This and all future requests should be directed to the given \verb!URI!.

\subsection{\texttt{302}: Found}
This is an example of industry practice contradicting the standard. The \verb!HTTP/1.0! specification (RFC 1945) required the client to perform a temporary redirect (the original describing phrase was "Moved Temporarily"), but popular browsers implemented \verb!302! with the functionality of a \verb!303! See Other. Therefore, \verb!HTTP/1.1! added status codes \verb!303! and \verb!307! to distinguish between the two behaviours. However, some Web applications and frameworks use the \verb!302! status code as if it were the \verb!303!.

\subsection{\texttt{303}: See Other (since \texttt{HTTP/1.1})}
The response to the request can be found under another \verb!URI! using a \verb!GET! method. When received in response to a \verb!POST! (or \verb!PUT!/\verb!DELETE!), the client should presume that the server has received the data and should issue a redirect with a separate \verb!GET! message.

\subsection{\texttt{304}: Not Modified (RFC 7232)}
Indicates that the resource has not been modified since the version specified by the request headers If-Modified-Since or If-None-Match. In such case, there is no need to retransmit the resource since the client still has a previously-downloaded copy.

\subsection{\texttt{305}: Use Proxy (since \texttt{HTTP/1.1})}
The requested resource is available only through a proxy, the address for which is provided in the response. Many \verb!HTTP! clients (such as Mozilla and Internet Explorer) do not correctly handle responses with this status code, primarily for security reasons.

\subsection{\texttt{306}: Switch Proxy}
No longer used. Originally meant ``Subsequent requests should use the specified proxy.''

\subsection{\texttt{307}: Temporary Redirect (since \texttt{HTTP/1.1})}
n this case, the request should be repeated with another \verb!URI!; however, future requests should still use the original \verb!URI!. In contrast to how \verb!302! was historically implemented, the request method is not allowed to be changed when reissuing the original request. For example, a \verb!POST! request should be repeated using another \verb!POST! request.

\subsection{\texttt{308}: Permanent Redirect (RFC 7538)}
The request and all future requests should be repeated using another \verb!URI!. \verb!307! and \verb!308! parallel the behaviours of \verb!302! and \verb!301!, but do not allow the HTTP method to change. So, for example, submitting a form to a permanently redirected resource may continue smoothly.






\section{\texttt{4XX}\emdash: Client Error}
The \verb!4XX! class of status code is intended for situations in which the client seems to have erred. Except when responding to a \verb!HEAD! request, the server should include an entity containing an explanation of the error situation, and whether it is a temporary or permanent condition. These status codes are applicable to any request method. User agents should display any included entity to the user.

\subsection{\texttt{400}: Bad Request}
The server cannot or will not process the request due to an apparent client error (e.g., malformed request syntax, invalid request message framing, or deceptive request routing).

\subsection{\texttt{401}: Unauthorized (RFC 7235)}
Similar to \verb!403! Forbidden, but specifically for use when authentication is required and has failed or has not yet been provided. The response must include a \verb!WWW!-Authenticate header field containing a challenge applicable to the requested resource. See Basic access authentication and Digest access authentication. \verb!401! semantically means "unauthenticated", i.e. the user does not have the necessary credentials.

{\scriptsize Note: Some sites issue \verb!HTTP! \verb!401! when an \verb!IP! address is banned from the website (usually the website domain) and that specific address is refused permission to access a website.}

\subsection{\texttt{402}: Payment Required}
Reserved for future use. The original intention was that this code might be used as part of some form of digital cash or micropayment scheme, but that has not happened, and this code is not usually used. Google Developers \verb!API! uses this status if a particular developer has exceeded the daily limit on requests.

\subsection{\texttt{403}: Forbidden}
The request was a valid request, but the server is refusing to respond to it. \verb!403! error semantically means ``unauthorized'', i.e. the user does not have the necessary permissions for the resource.

\subsection{\texttt{404}: Not Found}
The requested resource could not be found but may be available in the future. Subsequent requests by the client are permissible.

\subsection{\texttt{405}: Method Not Allowed}
A request method is not supported for the requested resource; for example, a \verb!GET! request on a form which requires data to be presented via \verb!POST!, or a \verb!PUT! request on a read-only resource.

\subsection{\texttt{406}: Not Acceptable}
The requested resource is capable of generating only content not acceptable according to the Accept headers sent in the request.

\subsection{\texttt{407}: Proxy Authentication Required (RFC 7235)}
The client must first authenticate itself with the proxy.

\subsection{\texttt{408}: Request Timeout}
The server timed out waiting for the request. According to \verb!HTTP! specifications: ``The client did not produce a request within the time that the server was prepared to wait. The client MAY repeat the request without modifications at any later time.''

\subsection{\texttt{409}: Conflict}
Indicates that the request could not be processed because of conflict in the request, such as an edit conflict between multiple simultaneous updates.

\subsection{\texttt{410}: Gone}
Indicates that the resource requested is no longer available and will not be available again. This should be used when a resource has been intentionally removed and the resource should be purged. Upon receiving a \verb!410! status code, the client should not request the resource in the future. Clients such as search engines should remove the resource from their indices. Most use cases do not require clients and search engines to purge the resource, and a ``\verb!404! Not Found'' may be used instead.

\subsection{\texttt{411}: Length Required}
The request did not specify the length of its content, which is required by the requested resource.

\subsection{\texttt{412}: Precondition Failed (RFC 7232)}
The server does not meet one of the preconditions that the requester put on the request.

\subsection{\texttt{413}: Payload Too Large (RFC 7231)}
The request is larger than the server is willing or able to process. Previously called ``Request Entity Too Large''.

\subsection{\texttt{414}: \texttt{URI} Too Long (RFC 7231)}
The \verb!URI! provided was too long for the server to process. Often the result of too much data being encoded as a query-string of a \verb!GET! request, in which case it should be converted to a \verb!POST! request. Called ``Request-\verb!URI! Too Long'' previously.

\subsection{\texttt{415}: Unsupported Media Type}
The request entity has a media type which the server or resource does not support. For example, the client uploads an image as image/\verb!svg+xml!, but the server requires that images use a different format.

\subsection{\texttt{416}: Range Not Satisfiable (RFC 7233)}
The client has asked for a portion of the file (byte serving), but the server cannot supply that portion. For example, if the client asked for a part of the file that lies beyond the end of the file. Called ``Requested Range Not Satisfiable'' previously.

\subsection{\texttt{417}: Expectation Failed}
The server cannot meet the requirements of the Expect request-header field.

\subsection{\texttt{418}: I'm a teapot (RFC 2324)}
This code was defined in 1998 as one of the traditional IETF April Fools' jokes, in RFC 2324, Hyper Text Coffee Pot Control Protocol, and is not expected to be implemented by actual HTTP servers. The RFC specifies this code should be returned by tea pots requested to brew coffee. This \verb!HTTP! status is used as an easter egg in some websites, including \verb!Google.com!.

\subsection{\texttt{421}: Misdirected Request (RFC 7540)}
The request was directed at a server that is not able to produce a response (for example because a connection reuse).

\subsection{\texttt{422}: Unprocessable Entity (WebDAV; RFC 4918)}
The request was well-formed but was unable to be followed due to semantic errors.

\subsection{\texttt{423}: Locked (WebDAV; RFC 4918)}
The resource that is being accessed is locked.

\subsection{\texttt{424}: Failed Dependency (WebDAV; RFC 4918)}
The request failed due to failure of a previous request (e.g., a \verb!PROPPATCH!).

\subsection{\texttt{426}: Upgrade Required}
The client should switch to a different protocol such as \verb!TLS/1.0!, given in the Upgrade header field.

\subsection{\texttt{428}: Precondition Required (RFC 6585)}
The origin server requires the request to be conditional. Intended to prevent ``the `lost update' problem, where a client \verb!GET!s a resource's state, modifies it, and \verb!PUT!s it back to the server, when meanwhile a third party has modified the state on the server, leading to a conflict.''

\subsection{\texttt{429}: Too Many Requests (RFC 6585)}
The user has sent too many requests in a given amount of time. Intended for use with rate limiting schemes.

\subsection{\texttt{431}: Request Header Fields Too Large (RFC 6585)}
The server is unwilling to process the request because either an individual header field, or all the header fields collectively, are too large.

\subsection{\texttt{451}: Unavailable For Legal Reasons}
A server operator has received a legal demand to deny access to a resource or to a set of resources that includes the requested resource. The code \verb!451! was chosen as a reference to the novel \emph{Fahrenheit 451}.






\section{\texttt{5XX}\emdash: Server Error}
The server failed to fulfil an apparently valid request.

Response status codes beginning with the digit \verb!5! indicate cases in which the server is aware that it has encountered an error or is otherwise incapable of performing the request. Except when responding to a \verb!HEAD! request, the server should include an entity containing an explanation of the error situation, and indicate whether it is a temporary or permanent condition. Likewise, user agents should display any included entity to the user. These response codes are applicable to any request method.

\subsection{\texttt{500}: Internal Server Error}
A generic error message, given when an unexpected condition was encountered and no more specific message is suitable.

\subsection{\texttt{501}: Not Implemented}
The server either does not recognise the request method, or it lacks the ability to fulfil the request. Usually this implies future availability (e.g., a new feature of a web-service \verb!API!).

\subsection{\texttt{502}: Bad Gateway}
The server was acting as a gateway or proxy and received an invalid response from the upstream server.

\subsection{\texttt{503}: Service Unavailable}
The server is currently unavailable (because it is overloaded or down for maintenance). Generally, this is a temporary state.

\subsection{\texttt{504}: Gateway Timeout}
The server was acting as a gateway or proxy and did not receive a timely response from the upstream server.

\subsection{\texttt{505}: \texttt{HTTP} Version Not Supported}
The server does not support the \verb!HTTP! protocol version used in the request.

\subsection{\texttt{506}: Variant Also Negotiates (RFC 2295)}
Transparent content negotiation for the request results in a circular reference.

\subsection{\texttt{507}: Insufficient Storage (WebDAV; RFC 4918)}
The server is unable to store the representation needed to complete the request.

\subsection{\texttt{508}: Loop Detected (WebDAV; RFC 5842)}
The server detected an infinite loop while processing the request (sent in lieu of \verb!208! Already Reported).

\subsection{\texttt{510}: Not Extended (RFC 2774)}
Further extensions to the request are required for the server to fulfil it.

\subsection{\texttt{511}: Network Authentication Required (RFC 6585)}
The client needs to authenticate to gain network access. Intended for use by intercepting proxies used to control access to the network (e.g., ``captive portals'' used to require agreement to Terms of Service before granting full Internet access via a Wi-Fi hotspot).




\section{Unofficial Codes}

The following codes are not specified by any RFC, but are used by third-party services to provide semantic or \verb!REST!ful error responses.

\subsection{\texttt{103}: Checkpoint}
Used in the resumable requests proposal to resume aborted \verb!PUT! or \verb!POST! requests.

\subsection{\texttt{420}: Method Failure (Spring Framework)}
A deprecated response used by the Spring Framework when a method has failed.

\subsection{\texttt{420}: Enhance Your Calm (Twitter)}
Returned by version 1 of the Twitter Search and Trends \verb!API! when the client is being rate limited; versions 1.1 and later use the \verb!429! Too Many Requests response code instead.

\subsection{\texttt{450}: Blocked by Windows Parental Controls (Microsoft)}
A Microsoft extension. This error is given when Windows Parental Controls are turned on and are blocking access to the given webpage.

\subsection{\texttt{498}: Invalid Token (Esri)}
Returned by ArcGIS for Server. A code of \verb!498! indicates an expired or otherwise invalid token.

\subsection{\texttt{499}: Token Required (Esri)}
Returned by ArcGIS for Server. A code of \verb!499! indicates that a token is required but was not submitted.

\subsection{\texttt{499}: Request has been forbidden by antivirus}
Produced by some programs such as \verb!wget! when a malicious site is intercepted.

\subsection{\texttt{509}: Bandwidth Limit Exceeded (Apache Web Server/cPanel)}
The server has exceeded the bandwidth specified by the server administrator; this is often used by shared hosting providers to limit the bandwidth of customers.

\subsection{\texttt{530}: Site is frozen}
Used by the Pantheon web platform to indicate a site that has been frozen due to inactivity

\subsection{Internet Information Services}
The Internet Information Services expands the \verb!4XX! error space to signal errors with the client's request.

\subsubsection{\texttt{440}: Login Timeout}
The client's session has expired and must log in again.

\subsubsection{\texttt{449}: Retry With}
The server cannot honour the request because the user has not provided the required information.

\subsubsection{\texttt{451}: Redirect}
Used in Exchange ActiveSync when either a more efficient server is available or the server cannot access the users' mailbox. The client is expected to re-run the \verb!HTTP! AutoDiscover operation to find a more appropriate server.


\subsection{\texttt{nginx}}
The \verb!nginx! web server software expands the \verb!4XX! error space to signal issues with the client's request.

\subsubsection{\texttt{444}: No Response}
Used to indicate that the server has returned no information to the client and closed the connection.

\subsubsection{\texttt{495}: \texttt{SSL} Certificate Error}
An expansion of the \verb!400! Bad Request response code, used when the client has provided an invalid client certificate.

\subsubsection{\texttt{496}: \texttt{SSL} Certificate Required}
An expansion of the \verb!400! Bad Request response code, used when a client certificate is required but not provided.

\subsubsection{\texttt{497}: \texttt{HTTP} Request Sent to \texttt{HTTPS} Port}
An expansion of the \verb!400! Bad Request response code, used when the client has made a \verb!HTTP! request to a port listening for \verb!HTTPS! requests.

\subsubsection{\texttt{499}: Client Closed Request}
Used when the client has closed the request before the server could send a response.

\subsection{CloudFare}
CloudFlare's reverse proxy service expands the \verb!5XX! error space to signal issues with the origin server.

\subsubsection{\texttt{520}: Unknown Error}
The \verb!520! error is used as a ``catch-all response for when the origin server returns something unexpected'', listing connection resets, large headers, and empty or invalid responses as common triggers.

\subsubsection{\texttt{521}: Web Server Is Down}
The origin server has refused the connection from CloudFlare.

\subsubsection{\texttt{522}: Connection Timed Out}
CloudFlare could not negotiate a \verb!TCP! handshake with the origin server.

\subsubsection{\texttt{523}: Origin Is Unreachable}
CloudFlare could not reach the origin server; for example, if the \verb!DNS! records for the origin server are incorrect.

\subsubsection{\texttt{524}: A Timeout Occurred}
CloudFlare was able to complete a \verb!TCP! connection to the origin server, but did not receive a timely \verb!HTTP! response.

\subsubsection{\texttt{525}: \texttt{SSL} Handshake Failed}
CloudFlare could not negotiate a \texttt{SSL}/\texttt{TLS} handshake with the origin server.

\subsubsection{\texttt{526}: Invalid \texttt{SSL} Certificate}
CloudFlare could not validate the \texttt{SSL}/\texttt{TLS} certificate that the origin server presented.



\rule{0.3\linewidth}{0.25pt}
\scriptsize

\href{https://www.webconfs.com/downloads/webconfs-status-codes.pdf}{\texttt{https://www.webconfs.com/}}
\href{https://github.com/jakewilliami?tab=overview}{\texttt{https://github.com/jakewilliami/}}


\end{multicols}
\end{document}
